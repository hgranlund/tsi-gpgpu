%!TEX root = ./main.tex
\chapter{Introduction}

In more recent years, the hardware developers have increased the performance of new systems by utilizing the power of parallel processing. The demand for better computer graphics has been a driving force for creating more powerful parallel computing hardware. Powerful graphics cards, which features a highly parallel architecture, is available in the consumer market to a relatively low price.

Unfortunately, this parallel power is not automatically utilized by traditional algorithms written with sequential execution in mind. In addition, parallelization is not a magic bullet. It can deliver blisteringly fast execution when the type of problem permits it, but for many classes of problems, benefiting form parallel processing capabilities is not trivial, or even possible.

Due to this need for different algorithms and specialized knowledge in order to harness the power of parallel execution, a lot of software is currently not benefiting from the possible performance of modern hardware. This is a point to be especially concerned with in engineering software, where complex, time consuming computations is commonplace.

In our thesis we have studied how to test the power of general processing of graphical processing units (GPGPU) in engineering software applications. We have studied this problem by developing faster, parallel algorithms for use in point cloud analysis software supplied by TechnoSoft Inc (TSI).

During the course of our work, we discovered that some of the assignment topics was not relevant in the way initially envisioned. At the same time, new possibilities arose that was not accounted for during the draft of the assignment text. In collaboration with our thesis advisors, we therefore choose to divert some from the original assignment.

An evaluation of GPGPU libraries like ViennaCL and cuBLAS was expected to be relevant for the thesis, but this proved to be entirely irrelevant for the particular problem we studied, since we ended up not being able to rely on any prefabricated GPGPU library. In addition, the grid smoothing algorithms for CFD analysis was not developed in time to be a part of our work, so this part of the assignment was left out as well.

%TODO: Make test fit one page.

%Early on it became clear that TSI had a very concrete engineering algorithm problem, related to point cloud analysis. Since it would be mutually beneficial for us and TSI to focus more on in-depth parallelization of algorithms related to this problem, we choose, i

% The original assignment focused on a broader, more literary focused study of comparing GPGPU tools, libraries and general techniques, with the addition of testing some of the studied techniques on select engineering algorithms. By making the engineering algorithm problem supplied by TSI the focus of our thesis, the work became more rooted in the real world. Just describing possible solutions and techniques would no longer be an option, every algorithm had to be implemented, parallelized, optimized, and tested in such a way that it could be used in a real world application. Studying and comparing GPGPU tools, libraries and general techniques would still be important, but more driven by the demand on the real world problem, instead of being an attempt of giving an exhaustive comparison of available GPGPU resources for engineering applications.

% This change of focus does probably mean that we have lost some of the scope originally intended, but we believe strongly that the depth and quality of the results we have been able to achieve, by focusing on building and implementing algorithms for a real world problem, is far superior than what would have been possible to achieve by focusing on the more broad, literary approach.

% Let us take a moment and go through how the changes noted above affect the original assignment text.

% The original text specified that the assignment would include a study of how to best utilize the high capacity for parallel processing. As a general topic, this is covered in short as a part of the background section. More so, this topic is covered indirectly as part of our work with the different algorithms for the problem supplied by TSI\@. For every step of development, this problem is addressed in direct relation to the current algorithm. This means that our coverage of this topic is more specific for our problem, but we believe that many of the techniques used in our implementations could be used on other problems.


% This could be an interesting topic to study related to a problem where such libraries could be applied, but in our work, a comparison of GPGPU libraries ended up being left out.

