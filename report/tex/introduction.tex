%!TEX root = ./main.tex
\chapter{Introduction}

In more recent years, the hardware developers have increased the performance of new systems, by utilizing the power of parallel processing. The demand for better computer graphics has been a driving force for creating more powerful parallel computing hardware. Powerful graphics cards, which features a highly parallel architecture, is today available in the consumer market to a relatively low price.

Unfortunately, this parallel power is not automatically utilized by traditional algorithms written with sequential execution in mind. In addition, parallelization is not a magic bullet. It can deliver blisteringly fast execution when the type of problem permits it, but for many classes of problems, benefiting from parallel processing capabilities, is not trivial, or even possible.

Due to this need for different algorithms and specialized knowledge, in order to harness the power of parallel execution, a lot of software is currently not benefiting from the possible performance of modern hardware. This is a point to be concerned with, in relation to engineering software, where complex, time consuming computations is commonplace.

In our thesis, we have studied how to utilize the power of general processing of graphical processing units (GPGPU) in engineering software applications. We have studied this problem, by developing faster, parallel algorithms, for use in point cloud analysis software, made by TechnoSoft Inc (TSI).

During the course of our work, we discovered that some of the assignment topics was not relevant, in the way initially envisioned. At the same time, new possibilities arose, that was not accounted for, during the draft of the assignment text. In collaboration with our thesis advisors, we therefore choose to divert some from the original assignment.

An evaluation of GPGPU libraries like ViennaCL and cuBLAS was expected to be relevant for the thesis, but this proved to be entirely irrelevant for the particular problem we studied, since we ended up not being able to rely on any prefabricated GPGPU library. In addition, the grid smoothing algorithms for CFD analysis was not developed in time to be a part of our work, so this part of the assignment was left out as well.
%TODO: Make test fit one page.