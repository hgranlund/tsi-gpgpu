%!TEX root = ./main.tex


\usepackage{setspace}
\usepackage{graphicx}
\usepackage{amssymb}
\usepackage{mathrsfs}
\usepackage{amsthm}
\usepackage{amsmath}
\usepackage{color}
\usepackage[Lenny]{fncychap}
\usepackage[pdftex,bookmarks=true]{hyperref}
\usepackage[pdftex]{hyperref}
\hypersetup{
    colorlinks,%
    citecolor=black,%
    filecolor=black,%
    linkcolor=black,%
    urlcolor=black
}
\usepackage[font=small,labelfont=bf]{caption}
\usepackage{fancyhdr}
\usepackage{times}
\usepackage[intoc]{nomencl}
\renewcommand{\nomname}{List of Abbreviations}
\makenomenclature
\usepackage{natbib}
\usepackage{float}
\floatstyle{boxed}
\restylefloat{figure}

% \usepackage[number=none]{glossary}
% \makeglossary
% \newglossarytype[abr]{abbr}{abt}{abl}
% \newglossarytype[alg]{acronyms}{acr}{acn}
\newcommand{\abbrname}{Abbreviations}
\newcommand{\shortabbrname}{Abbreviations}
%\makeabbr
\newcommand{\HRule}{\rule{\linewidth}{0.5mm}}

\renewcommand*\contentsname{Table of Contents}


% Define aothor and title
\newcommand{\myauthorA}{Simen Haugerud Granlund}
\newcommand{\myauthorB}{Teodor Ande Elstad}
\newcommand{\mytitle}{Utilizing general-purpose computing on graphic processing units (gpgpu) for engineering algorithm speed up}

\pagestyle{fancy}
\fancyhf{}
    \fancyfoot[LO]{\mytitle} %Title of paper
    \fancyfoot[RE]{Stud.techn. \myauthorA, Stud.techn. \myauthorB} % Name of author
    \fancyfoot[LE,RO]{\thepage} % Print page number
\renewcommand{\chaptermark}[1]{\markboth{\chaptername\ \thechapter.\ #1}{}}
\renewcommand{\sectionmark}[1]{\markright{\thesection\ #1}}
\renewcommand{\headrulewidth}{0.1ex}
\renewcommand{\footrulewidth}{0.1ex}
\fancypagestyle{plain}{\fancyhf{}\fancyfoot[LE,RO]{\thepage}\renewcommand{\headrulewidth}{0ex}}


%To inclode source code filed
\usepackage{listings}

% Define reserch question label
\newtheorem{myrq}{RQ}
\usepackage[parfill]{parskip}

% Define definition label
% \theoremstyle{def}
% \newtheorem{mydef}{Definition}


%to include pdf
\usepackage{pdfpages}




% from  old version

% \documentclass[a4paper, english, 12pt, twoside, titlepage]{article} %two-sided
% \usepackage{babel, fancyhdr, amsmath, amsfonts, amssymb, graphicx, float, enumerate, hyperref}
% \usepackage[utf8]{inputenc}
% \usepackage[portrait,pdftex]{geometry}
% %Theorem and definitions formatting
% \usepackage{amsthm}

% %prettier references
% \usepackage{natbib}

% \usepackage[autostyle]{csquotes}

% % To use appendix envoronment
% \usepackage[title,titletoc]{appendix}
% %nomenclature
% \usepackage{nomencl}
% \makenomenclature
% \renewcommand{\nomname}{List of abbreviations and definitions}

%used to get  figure, table and equation-numbers show the section they belong to, like ``Figure 2.1``
\numberwithin{figure}{section}
\numberwithin{table}{section}
\numberwithin{equation}{section}

% %nicer captionsv
% \usepackage[font=small,format=plain,labelfont=bf,up,textfont=it,up]{caption}
% \usepackage{fancyhdr}
% \setlength{\headheight}{15pt}

% % Regluar style, with top and bottom rulers, section name at top
% % and title and page number at bottom, set up for two-sided printing
% \fancypagestyle{mystyle}{
%     \fancyhf{} % remove everything
%     \fancyhead[RO,LE]{\nouppercase\leftmark} %section name in lowercase
%     \fancyfoot[LO]{\mytitle} %Title of paper
    % \fancyfoot[RE]{Stud.techn. \myauthorA, Stud.techn. \myauthorB} % Name of author
%     \fancyfoot[LE,RO]{\thepage} % Print page number
%     \renewcommand{\headrulewidth}{0.5pt} % Ruler at top
%     \renewcommand{\footrulewidth}{0.5pt}% Ruler at boottom
%     \addtolength{\headheight}{15pt} % make space for the rule
% }

% % Plain style, just page number, set up for two-sided printing
% \fancypagestyle{plain}{ %
%     \fancyhead{} % get rid of headers on plain pages
%     \renewcommand{\headrulewidth}{0pt} % and the line
%     \fancyfoot{} % get rid of footers on plain pages
%     \renewcommand{\footrulewidth}{0pt} % and the line
%     \fancyfoot[LE,RO]{\thepage} % Print page number
% }


% % this package adds clickable links to http addresses and footnotes
% \usepackage{hyperref}

% %set links to be painted in black
% \hypersetup{colorlinks,
% linkcolor=black,
% filecolor=black,
% urlcolor=black,
% citecolor=black}

% this package makes url's breakable with hyperref (but loses the cyan boxes)
% \usepackage{breakurl}
\usepackage{multirow}
\usepackage{subfigure}
% Change margins for two-sided
% \oddsidemargin 1.0cm
% \evensidemargin 0.0cm


%Style to definition and RQ's
\newtheoremstyle{def}% name
  {.5\baselineskip\@plus.2\baselineskip\@minus.2\baselineskip}% Space above
  {.5\baselineskip\@plus.2\baselineskip\@minus.2\baselineskip}% Space below
  {\slshape}% Body font
  {\parindent}%Indent amount (empty = no indent, \parindent = para indent)
  {\bfseries}%  Thm head font
  {.}%       Punctuation after thm head
  { }%      Space after thm head: " " = normal interword space;
        %       \newline = linebreak
  {}%       Thm head spec


\theoremstyle{def}
% Define definition label
\newtheorem{mydef}{Definition}




% %---------------------------------------------------------
