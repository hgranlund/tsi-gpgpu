%!TEX root = ./main.tex
\usepackage{setspace}
\usepackage{graphicx}
\usepackage{amssymb}
\usepackage{mathrsfs}
\usepackage{textcomp, amssymb}  % additional symbols (there are more packages)
\usepackage{amsthm}
\usepackage{amsmath}
\usepackage{color}
\usepackage[Lenny]{fncychap}
\usepackage[pdftex,bookmarks=true]{hyperref}
\usepackage[pdftex]{hyperref}
\hypersetup{
    colorlinks,%
    citecolor=black,%
    filecolor=black,%
    linkcolor=black,%
    urlcolor=black
}
\usepackage[font=small,labelfont=bf]{caption}
\usepackage{fancyhdr}
\usepackage{times}
\usepackage[intoc]{nomencl}
\renewcommand{\nomname}{List of Abbreviations}
\makenomenclature
\usepackage{natbib}
\usepackage{float}
\restylefloat{figure}


\newcommand{\abbrname}{Abbreviations}
\newcommand{\shortabbrname}{Abbreviations}
%\makeabbr
\newcommand{\HRule}{\rule{\linewidth}{0.5mm}}
\renewcommand*\contentsname{Table of Contents}

% Define aothor and title
\newcommand{\myauthorA}{Simen Haugerud Granlund}
\newcommand{\myauthorB}{Teodor Ande Elstad}
\newcommand{\mytitle}{Utilizing general-purpose computing on graphic processing units for engineering algorithm speed up}

\pagestyle{fancy}
\fancyhf{}
    \fancyfoot[LO]{\mytitle} %Title of paper
    \fancyfoot[RE]{Stud.techn. \myauthorA, Stud.techn. \myauthorB} % Name of author
    \fancyfoot[LE,RO]{\thepage} % Print page number
\renewcommand{\chaptermark}[1]{\markboth{\chaptername\ \thechapter.\ #1}{}}
\renewcommand{\sectionmark}[1]{\markright{\thesection\ #1}}
\renewcommand{\headrulewidth}{0.1ex}
\renewcommand{\footrulewidth}{0.1ex}
\fancypagestyle{plain}{\fancyhf{}\fancyfoot[LE,RO]{\thepage}\renewcommand{\headrulewidth}{0ex}}

%To inclode source code filed
\usepackage{listings}

\usepackage[parfill]{parskip}
\usepackage[title,titletoc]{appendix}
%to include pdf
\usepackage{pdfpages}

%used to get  figure, table and equation-numbers show the section they belong to, like ``Figure 2.1``
\numberwithin{figure}{section}
\numberwithin{table}{section}
\numberwithin{equation}{section}

%Style to definition and RQ's
\newtheoremstyle{def}% name
  {.5\baselineskip\@plus.2\baselineskip\@minus.2\baselineskip}% Space above
  {.5\baselineskip\@plus.2\baselineskip\@minus.2\baselineskip}% Space below
  {\slshape}% Body font
  {\parindent}%Indent amount (empty = no indent, \parindent = para indent) 
  {\bfseries}%  Thm head font
  {.}%       Punctuation after thm head
  { }%      Space after thm head: " " = normal interword space;
        %       \newline = linebreak
  {}%       Thm head spec

\theoremstyle{def}
\newtheorem{mydef}{Definition}

% Define reserch question label
\newtheorem{myrq}{RQ}
% Define a big-O  notation command

\newcommand{\BigO}[1]{\ensuremath{\operatorname{O}\bigl(#1\bigr)}}

\usepackage[chapter]{algorithm}
\usepackage{algpseudocode}
\renewcommand{\algorithmicrequire}{\textbf{Input:}}
\renewcommand{\algorithmicensure}{\textbf{Output:}}
\usepackage{numprint}
% %---------------------------------------------------------
