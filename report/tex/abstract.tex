%!TEX root = ./main.tex
\begin{center}
{\Large\bfseries \mytitle}
\end{center}

\begin{center}
{\bfseries\slshape Project assignment TDT4560}
%\end{center}
\\[1.0cm]
{\bfseries\slshape
Stud.techn. \myauthorA, Stud.techn. \myauthorB \\}
Department of Engineering Design and Materials \\
Faculty of Engineering Science and Technology\\
Norwegian University of Science and Technology
\end{center}
\section*{Abstract}

The goal of this thesis, is to investigate the possible benefits of using general-purpose computing on graphics processing units (GPGPU), in order to speed up the execution of calculations in engineering applications.

The thesis focuses primarily on speeding up the process of analyzing laser scan point clouds, in software developed by TechnoSoft Inc. This is achieved by developing faster algorithms for solving the k-nearest neighbors (kNN), and All-kNN problem, optimized for point cloud data.

A parallel brute-force algorithm is developed, which is capable of solving the kNN problem up to $70$ times faster than a similar algorithm developed by Garcia et.al \cite{Garcia2008}, when working on comparable data.

By utilizing the k-d tree data structure, a parallel tree-build and query algorithm is developed, suitable for solving the All-kNN problem. This algorithm improves the result obtained by the brute-force algorithm by up to $300$ times.

% Norsk sammendrag:

% Målet med denne avhandlingen, er å utforske mulighetene knyttet til å anvende GPGPU (general-purpose computing on graphics processing units) for å forbedre ytelsen til tunge beregninger i programvarebaserte ingeniørverktøy.

% Avhandlingen tar for seg en problemstilling, knyttet til analyse av punktskyer, i programvare utviklet av TechnoSoft Inc. Ytelsen i denne programvaren blir forsøkt økt, ved å utvikle raskere algoritmer, for løsning av kNN (k nærmeste naboer) og Alle-kNN problemet, optimalisert for punktskybaserte data.

% En GPU-parallellisert “brute-force” algoritme blir utvikle,t som er i stand til å løse kNN problemet $70$ ganger raksere enn en tilsvarende algoritme, utviklet av Garcia et.al. \cite{Garcia2008}, anvendt på tilsvarende data.

% Ved å anvende k-d trær, en GPU-parallelisert algoritme blir utviklet, egnet for å løse Alle-kNN problemet. Denne algoritmen forbedrer resultatet oppnådd gjennom bruk av “brute-force algoritmen med 300 ganger.


\clearpage\
