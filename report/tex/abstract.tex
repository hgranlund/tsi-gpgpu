%!TEX root = ./main.tex
\begin{center}
{\Large\bfseries \mytitle}
\end{center}

\begin{center}
{\bfseries\slshape Project assignment TDT4560}
%\end{center}
\\[1.0cm]
{\bfseries\slshape
Stud.techn. \myauthorA, Stud.techn. \myauthorB \\}
Department of Engineering Design and Materials \\
Faculty of Engineering Science and Technology\\
Norwegian University of Science and Technology
\end{center}
\section*{Abstract}

The goal of this thesis, is to investigate the possible benefits of using general-purpose computing on graphics processing units (GPGPU), in order to speed up the execution of calculations in engineering applications.

The thesis focuses mainly on speeding up processing of laser scan point clouds in software developed by TechnoSoft Inc, by reducing the time needed to perform a k-nearest neighbors search for every points in the point cloud, for k significantly smaller than N.

First, a brute force based approach, based on the work by Vincent Garcia, is refined and evaluated for this application. Although a large speedup is gained by using this approach, the overall performance is to low to be used on the problem sizes required by TechnoSoft Inc. Second, a kd-tree based approach is investigated, and the inherent difficulties of parallelizing this tree based algorithm is addressed. Finally, it is shown that the GPGPU based algorithms developed in this thesis can be used to speed up a k-nearest neighbors search in a large point cloud.
