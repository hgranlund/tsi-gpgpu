%!TEX root = ./main.tex

\chapter{The quest for a faster kNN search} % (fold)
\label{sec:the_quest_for_a_faster_knn_search}

\section{A short evaluation of OpenCL and CUDA} % (fold)
\label{sub:a_short_evaluation_of_opencl_and_cuda}

% section a_short_evaluation_of_opencl_and_cuda (end)

\section{Investigation of a brute force approach based on Garcia} % (fold)
\label{sub:investigation_of_a_brute_force_approach_based_on_garcia}

% section investigation_of_a_brute_force_approach_based_on_garcia (end)

% Possible RQ for the k-d tree section.
\begin{myrq}
\label{rq:structure_aml}
    It is possible to use a k-d tree, to increase the performance of a large number of repeated, 3-d, low k queries in a large dataset.
\end{myrq}


\section{Application of k-d trees to the kNN problem} % (fold)
\label{sub:application_of_kd_trees_to_the_knn_problem}

A usual strategy, when wanting to improve the performance of repeated queries in a large dataset, is to organize the dataset into some data structure especially suited for fast querying. When choosing this strategy, you are trading of the additional time penalty of building the data structure, for increased performance on each query, hopefully offsetting the build-time penalty. When considering the needs of TechnoSoft Inc, as described in section \ref{a_short_introduction_to_the_kNN_algorithm}, this strategy seems like a good fit.

In this section, we will present the k-d tree data structure, and show how it can be used for operating on three-dimensional point cloud data. Then an investigation is performed, in order to determine the possible benefits of a parallel k-d tree based algorithm.


\subsection{Using k-d trees for point cloud data} % (fold)
\label{ssub:using_k_d_trees_for_point_cloud_data}

A k-d tree can be thought of as a binary search tree in k dimensions.

A few different variations exist, but we will focus our explanation around a 2D example, storing point data in all nodes. The plane is split into two sub-planes along one of the axis (in our example the y-axis) and all the nodes are sorted as to whether they belong to the left or right of this split. To determine the left and right child of the root node, the two sub-planes are again split at an arbitrary point, this time cycling to the next axis (in our example the x-axis) and the

\begin{figure}[ht!]
\centering
\includegraphics[width=120mm]{../gfx/Kdtree_2d.png}

\caption{2D KD-tree}
\label{fig:kdtree_2d}
\end{figure}

In order to build a k-d tree for 3D space, you simply cycle through the three dimensions, instead of two.

Given the previous splits and selection of nodes, the resulting binary tree would be as shown in the illustration under. (All illustrations gratuitously borrowed from
%[Wikipedia](http://en.wikipedia.org/wiki/k-d_tree))

\begin{figure}[ht!]
\centering
\includegraphics[width=120mm]{../gfx/Tree_0001.png}

\caption{corresponding-binary-tree}
\label{fig:tree_0001}
\end{figure}

Given that the resulting binary tree is balanced, we get an average search time for the closest neighbor in O(log2(n)) time. For values of k << n, the same average search time can be achieved, with minimal changes to the algorithm, when searching for the k closest neighbors. It is known from literature that balancing the tree can be achieved by always splitting on the meridian node. Building a k-d tree in this manner takes O(kn log2(n)) time.

Interested readers is encouraged to look at the paper Multidimensional binary search trees used for associative searching by Jon Louis Bentley, where k-d trees first was described.


% subsection using_k_d_trees_for_point_cloud_data (end)



\subsection{The serial base algorithm} % (fold)
\label{ssub:the_serial_base_algorithm}

\begin{enumerate}
    \item Build a balanced k-d tree from the point cloud.
    \item Query the tree for different sets of neighbors.
\end{enumerate}

Time complexity:

Steps:
\begin{enumerate}
    \item O(n log2 n). Achieving this speed is dependent on an efficient algorithm for finding the meridian.
    \item Approximately O(log2 n), but dependent on size of k.
\end{enumerate}

% section application_of_kd_trees_to_the_knn_problem (end)
\begin{algorithm}                      % enter the algorithm environment
\caption{Seriel Tree Build thing}          % give the algorithm a caption
\label{alg:seriel_tree_build}                           % and a label for \ref{} commands later in the document
\begin{algorithmic}                    % enter the algorithmic environment
    \REQUIRE $n \geq 0 \vee x \neq 0$
    \ENSURE $y = x^n$
    \STATE $y \Leftarrow 1$
    \IF{$n < 0$}
        \STATE $X \Leftarrow 1 / x$
        \STATE $N \Leftarrow -n$
    \ELSE
        \STATE $X \Leftarrow x$
        \STATE $N \Leftarrow n$
    \ENDIF
    \WHILE{$N \neq 0$}
        \IF{$N$ is even}
            \STATE $X \Leftarrow X \times X$
            \STATE $N \Leftarrow N / 2$
        \ELSE[$N$ is odd]
            \STATE $y \Leftarrow y \times X$
            \STATE $N \Leftarrow N - 1$
        \ENDIF
    \ENDWHILE
\end{algorithmic}
\end{algorithm}


\section{Development of a parallel kd-tree build algorithm} % (fold)
\label{sub:development_of_a_parallel_kd_tree_build_algorithm}
The build process is by far the most expensive operation on a k-d tree, and parallelizing it could reduce the overall runtime significantly, when solving the kNN problem. This is not as straight-forward as it might seem. The serial k-d tree build algorithm is usually implemented as a recursive function, since recursive functions tend to go along well with tree-based data structures. On the other hand, performance in CUDA is based on efficient use of a massive number of lightweight threads, and to get a fast algorithm one have to split the work between as many threads as possible. This is only possible if it is easy to divide the work into independent subtasks, where data communication is kept at a minimum. In a recursive context, execution flow in hidden inside each threads call stack. Information needed by other threads is therefore not easy to obtain. To solve this, a iterative approach should be used, which can be implemented with a global accessible execution flow. This will however, give a more complex k-d tree build algorithm, and it is not certain that this algorithm is easy to parallelize. 

\begin{myrq}
\label{rq:parallel_build}
    It is possible to parallelize the k-d tree build algorithm, in such a way that it gives a significant speed improvement compared to the serial algorithm.
\end{myrq}

In order to investigate RQ~\ref{rq:parallel_build}, we have to look a bit closer at the different steps of the k-d tree build algorithm and investigate different parallelization strategies.

\subsection{From recursive to iterative implementation} % (fold)
\label{ssub:from_recursive_to_iterative_implementation}

Before we dive into the parallelization strategy and how the parallelization can be done, lets try to make a iterative solution. We can start by enumerating the different steps in the recursive implementation.

\begin{enumerate}
    \item Find the median of the points along a specified axis. This median point becomes the value of the current node.
    \item Sort all points with lower values than the median to the left of the median, and all the points with higher values than the median to the right.
    \item Perform this algorithm recursively on the left and right set of nodes.
\end{enumerate}

From the steps one can see that for each node in the k-d tree, one have to partition a list around it's median. We will call this operation Balance-Subtree. If we analyze the Algorithm~\ref{alg:seriel_tree_build}, we see that there are two recursive calls. This is logical, because we are building a binary tree where each node have two children. The interesting observation is that a node balance is only dependent on the parent node. This means that each tree level are independent and can be done iteratively.

The k-d tree construction basically boils down to successively balance each node in the tree. This leads to a basic reimplementation, see Algorithm~\ref{alg:iterativ_tree_build}. It goes through each level of the tree, starting at the top, and balances each node successively down the tree.

\begin{algorithm}
\caption{Iterative k-d tree build}
\label{alg:iterativ_tree_build}
\begin{algorithmic}
    \Require{An array of points, $T$}
    \Ensure{$T$, as a k-d formated array}
    \Function{Build-KD-Tree}{$T$}
        \ForAll{$L \in \{\text{all levels in } T\}$}
            \ForAll{$S \in L$}
                \State$d \gets |L| \bmod k$ \Comment{k = 3 for a three dimensional k-d tree}
                \State \Call{Balance-Subtree}{$S$, $d$}
            \EndFor
        \EndFor
    \EndFunction
\end{algorithmic}
\end{algorithm}
% subsection from_recursive_to_iterative_implementation (end)

\subsection{Parallelization strategy} % (fold)
\label{ssub:parallelization_strategy}

Now that an iterative solution have been created, lets start looking at how this algorithm might be parallelized. First a good overall parallelization strategy has to be found. A good strategy manage to easily split the main task into small individual subtasks, that can be performed in parallel, while maintaining a minimal need for communication between the subtasks.

When we converted our k-d tree algorithm from a recursive to a iterative solution some interesting observations was made. One observation is that a node balance is only dependent on the parent node. This means that each tree level are independent, which acts as a good start for our parallelization strategy.

This also implies that all subtrees in the k-d tree generation are independent. Hence, the tree corresponding to the left and right child of a node can be done in parallel without any communication. 

The data is also independent, as a result of how we represent the tree as an array. By data independent, it is meant that the data structure easily can be partitioned to each subtask. In our case this will be to successively partition the tree into contentious subtrees.

These observation implies that we can divide the tree levels into dependent tasks, where each node balance in a tree level is a independent subtask. This gives us an power of two increasing number of parallel tasks as we go down a tree level. The subtree size will decrease with a factor of two in each downward step, see Fig~\ref{fig:tree_level_development}

This parallelization strategy gives many concurrent operations at the lower level of the tree, but at the initial levels, will hardly get any parallelization at all. To compensate for this one could seek to parallelize the work done in each Balance-Subtree procedure, which also act as a parallelizations strategy.

Both strategies can be used in conjunction with each other. The parallel balance node task algorithm can be used to speed up the early iterations, where the amount of nodes in a tree level is small. As well as further parallelize the subtasks in later tree level iterations. This strategy also fit well to our choice of tree representation. One parallel operation can now take the tree representation, split it into subtrees, and balance each one. 

\begin{figure}[ht!]
\centering
\includegraphics[width=100mm]{../gfx/Tree_level_development.png}

\caption{Development of subtasks as the kd-tree generation progresses. It shows, at each tree level, how many nodes there is to parallelize and how big each node balancing is.}
\label{fig:tree_level_development}
\end{figure}
% subsection parallelization_strategy (end)

\subsubsection{CUDA parallelization} % (fold)
\label{ssub:divide_work_on_cuda}

CUDA have a special architecture that should be taken into account when parallelizing an algorithm. To efficiently use CUDA, the program has to keep thousands of threads occupied, otherwise the benefit of CUDA disappears. The CUDA programming model is build up by a grid if independent block, i.e.\ execution can not be synchronized across blocks. Execution can only be synchronized between the 1 to 1024 theoretical threads launched inside a block~\cite{cuda_programming_guide}. Thread synchronization is important when multiple threads cooperate on one task, because at some point information has to be exchanged.

Our parallelization strategy states that we have to balance one tree level after another, since they are dependent. This implies that the threads need to communicate between each tree level. One CUDA kernel should therefore balance a complete tree level. The other alternative would be to build the hole tree in one block, which would restrict our kernel to only be executed on one SM\@.

The next step is to split the tree level balance between the CUDA resources. The number of node in a tree level increases with the power of two, as we go down the tree. Fig~\ref{fig:tree_level_development} shows that our kernel, the tree balance, changes throughout the build process. First only one node needs to be balanced, e.g only one parallel operation. At the end there are $m$ different nodes to work on. The problem size also changes, at the top, it is $m$ per node and goes towards $1$, as the tree level increases.

This varying problem sizes and subtasks, makes it hard to create a good work distribution between the CUDA resources. At the top part of the tree it is optimal to use many blocks to balance a node, but at the end, it is desirable to balance many nodes inside a block. We choose a middle ground, to balance one node in one block. This means that there should be an overall good performance with a peek at the middle three levels.
% subsection divide_work_on_cuda (end)

\subsection{Parallelization of Subtree-Balance} % (fold)
\label{ssub:selecting_a_algorithm}

With the overall parallelization strategy planned, we can start investigate the most time-consuming operation, balancing a tree level. We have already determent that the parallelization should be done in one block, which means that one operation is done per SM\@. In other words, the task can potentially be parallelized between 1024 theoretical threads. Lets start investigating different approaches.

The main operation is to find a median. As we have seen in Section~\ref{sub:application_of_kd_trees_to_the_knn_problem}, many algorithms for finding median exist. Since we now want to implement the algorithm with CUDA, the environment has changed, and quick-select may not be the best alternative anymore. The first problem with quick-select is that it is recursive, which makes it hard to parallelize on CUDA\@. Therefore it may be profitable to look at other, more parallelization friendly, algorithms.

First reusing the bitonic sort was investigated. Given a sorted list one can find the median directly, by simply looking at the midmost element of the array. The partitioning is also done in the process. Unfortunately this strategy proved unsuccessful, as re-purposing the bitonic algorithm for such an task proved difficult. The reason for this is that a pure bitonic sorting network only manage to sort lists with a length of power of two. The normal solution is to create a longer list then needed, but wasting this much memory on on the GPU is not a very good solution. 

Another way to make bitonic sort work with a list of any size is described by K.E. Batcher~\cite{Batcher:1968}. The description of this solution is very lengthy, and has been omitted, since it introduces a lot of divergence, that would be decremental to performance on the GPU\@. We also have the inherent downside of sorting a list in order to find the median, since \BigO{m} algorithms for finding the median exist, compared to the \BigO{m log(m)} time required by sorting.

%TODO: Check agglomeration vs adversarial
Bucket-sort and radix-sort based algorithms are investigated in the paper Fast K-selection Algorithms for Graphics Processing Units by Alabi et.al\@.~\citep{Alabi:2012}. The big difference between them is the constant time penalty. The radix sort have a more exact time complexity of \BigO{b\ m}, where b is the number of bits in each number. While the penalty for bucket select is \BigO{a\ m}, where $a$ denotes the degree of agglomeration in the values. In other words, the algorithm is week when the points are clusters together. His results shows that bucket select normally is slightly faster, except when $a$ is high. Although bucket select normally have better results, we expect a high degree of agglomeration in our application, so we choose radix select.
% subsubsection selecting_a_algorithm (end)

\subsubsection{Radix select} % (fold)
\label{ssub:radix_select}

The radix select is based on a bitwise partitioning, much like radix sort~\cite[Chapter 8.3]{Cormen:2001}. In each step, elements are partitioned in two subgroups based on the current bit. Then the subgroup that contains the median is determined, and the search continue in that subgroup until the median is found.

\begin{figure}[ht!]
\centering
\includegraphics[width=100mm]{../gfx/Radix_select.png}

\caption{An illustration of radix selection~\cite{cayman:2012}.}
\label{fig:radix_select}
\end{figure}

When it comes to create a parallelization strategy for radix select it is first advisable to take a look at a highly optimized radix sort, like the variant introduced by Merrill~\cite{MerrillG11}. The radix select can easily be reduced from a radix sort, and many concepts can therefore be reused. An other interesting implementation is the radix select by Alabi~\citep{Alabi:2012}. They both uses a parallelization strategy by splitting each radix partition into parallel operations. The way our and Alabi's solution differs from Merrill's solution, is that we start on the most significant digit, since a least significant digit approach will not reduce the k'th order statistic problem in each step.
% subsection radix_select (end)

\subsubsection{Our implementation} % (fold)
\label{ssub:our_implementation}

% TODO: Ta med bigO kalkulasjon på hele algorithmen.
Our implementation is based on Algorithm~\ref{alg:iterativ_tree_build}. Calling Balance-Subtree on all nodes in a tree-level, is performed on the GPU as a CUDA kernel. The outer for-loop is executed on the CPU, and launches the kernel for increasing tree-levels, until the entire tree is built. The complete implementation can be found in Appendix~\ref{sec:paralell_k_d_tree_build}.

Algorithm~\ref{alg:parallel_balance_subtree} is parallelized only within a single CUDA block. This means that the parallel threads are able to communicate. The algorithm is based around a repeat-until loop, which basically do all the work. The loop keeps track of a partition array. This array is then sliced in to, by giving each thread a portion of the array, each thread then counts how many zeros it finds in the current bit position. The cut, as the arrows in Figure~\ref{fig:radix_select} shows, is calculated by doing a reduce sum operation~\citep{parallel_reduction_in_cuda}. This is repeated until the partition contains the median, which is when every bit is used or when the partition size is one. After the loop, the array is transformed. Such that the median is in the center, with lesser elements on the left and bigger on the right.

\begin{algorithm}[ht]
\caption{Parallel subtree balance}
\label{alg:parallel_balance_subtree}
\begin{algorithmic}
    \Require{A subtree $S$ of length $m$, and dimension $d$}
    \Ensure{A balanced subtree, $S$}
    \Function{Balance-Subtree}{$T$, $d$}
        \State Let $l$ and $u$ be the upper and lower bond of current partition.
        \State Let $P$ be all nodes in $S$
        \Repeat
            \ForAll {$\{p \in P \mid l < p < u\}$}
                \State $Z(t) \gets $  \text{Occurrences of zeros in current bit, $b$, found by thread, $t$.}
            \EndFor
            \State $c \gets \Call{Sum-Reduce}{Z}$ \Comment{$c$ is the cut of the current partition $P$.}
            \If{$u-c \ge m/2$}
                \State$ u \gets u-c$
            \Else
                \State$l \gets u-c$
            \EndIf
            \State $b \gets b+1$ \Comment{Move to the next bit}
            \State \Call{Synchronize-Threads}{\ }
        \Until{$u-l<1 \lor b > \Call{Radix}{p \in P} $}
        \State \Call{Partition}{$S, P$}
    \EndFunction
\end{algorithmic}
\end{algorithm}

In CUDA thread instruction run sequentially in warps of 32. Control flow divergence within a warp can therefore significantly destroy the instruction throughput. This is because the different execution paths must be serialized, as all the thread in a warp share the same program counter~\cite{cuda_c_best_practices_guide}. The total number of instruction in a warp will therefore increase. Any conditional operator, e.g\@. if and switch, should be used with care, since it may branch the control flow. Optimizing the use of conditionals, in order to reduce the amount of branching in the program control flow, will give better performance.

In our implementation, all threads perform the same thing in every iteration. This will give a low thread branching. The loop is also almost done equal times by all thread, one time for each bit used to represent the points. The one $if$ statement in the iteration, can be reduced to only contain one statement, which make the divergent thread branch small. The code is therefore good in regard of divergence.

An other aspect to consider, is the CUDA memory hierarchy. It is beneficial to use the fastest suitable memory. In our case, this include the shared memory, which is a fast memory shared between all threads in a block. The downside is the memory size, it may not be enough space to store our subtree. Although we may not be able to store the hole subtree, there are some date we can put in the shared memory. The zero counter array, shown as $Z(t)$ in Algorithm~\ref{alg:parallel_balance_subtree}, is a perfect candidate. Every thread only use one integer and it is shared between all threads throughout the execution. It will therefore cause a big impoverishment.

The complete implementation can be found in Appendix~\ref{sec:multiple_radix_select}.
% subsection our_implementation (end)

\subsection{Further improvements} % (fold)
\label{ssub:further_development}

Let us take a step back and look at RQ~\ref{rq:parallel_build}. The possibility of parallelizing the build algorithm is achieved. Initial optimization has been performed, so according to the APOD design cycle, we should test our implementation.

\begin{figure}[ht!]
\centering
\includegraphics[width=100mm]{../gfx/the_jumps.png}

\caption{Timing results from a intermediate version of the parallel k-d tree build algorithm.}
\label{fig:gpuv1_vs_cpu}
\end{figure}

With the test setup as described in Section~\ref{sec:test_environment}, the current version of the algorithm gave results as shown in Figure~\ref{fig:gpuv1_vs_cpu}. The most interesting observation are the big jumps in the graph. If one look closely these jumps happens every time the problem size exceeds a power of two, as for example when the size passes $8388608$ the timing increases from $2703 ms$ to $4335 ms$. The results of further investigation is shown in Table~\ref{tbl:nvprc_balance_branch}. It shows how long time each tree level takes, and how the different tree level operations varies throughout the build process.

\begin{table}[ht]
\centering
    \begin{tabular}{|l|l|l|l|}
        \hline
        \textbf{Tree level} & \textbf{Time [ms]} & \textbf{Subtrees} & \textbf{Size}\\ \hline
        1          & 52       & 1                  & 1000000      \\ \hline
        2          & 26       & 2                  & 500000       \\ \hline
        3          & 13       & 4                  & 250000       \\ \hline
        4          & 8        & 8                  & 125000       \\ \hline
        5          & 7        & 16                 & 62500        \\ \hline
        6          & 6        & 32                 & 31250        \\ \hline
        7          & 6        & 64                 & 15625        \\ \hline
        8          & 6        & 128                & 7812         \\ \hline
        9          & 7        & 256                & 3906         \\ \hline
        10         & 7        & 512                & 1953         \\ \hline
        11         & 8        & 1024               & 976          \\ \hline
        12         & 10       & 2048               & 488          \\ \hline
        13         & 16       & 4096               & 244          \\ \hline
        14         & 26       & 8192               & 122          \\ \hline
        15         & 52       & 16384              & 61           \\ \hline
        16         & 105      & 32768              & 30           \\ \hline
        17         & 202      & 65536              & 15           \\ \hline
        18         & 389      & 131072             & 7            \\ \hline
        19         & 768      & 262144             & 3            \\ \hline
    \end{tabular}
    \caption{Development of a k-d tree build with a million points, showing how the different tree level operations varies throughout the build process.}
    \label{tbl:nvprc_balance_branch}
\end{table}

The table reveal some weaknesses of our algorithm, that is based around how the CUDA resources was divided in Section~\ref{ssub:divide_work_on_cuda}. It performs badly when the problem size or the number of subtrees is relatively large. The potential for parallelizing the workload for the first and last iterations is not being fully utilized. This is due to the implementation forcing one version of the radix select algorithm to work on all problem types. This is not optimal for dividing CUDA resources, and as a result, we get high penalties when the problem reaches unsuitable values.

This hypothesis can also explain the big jumps in runtime. The observations above correlates perfectly with the tree hight, since the hight of a binary tree is the binary logarithm of the tree size. This implies that the jumps happens when an additional tree-level is required in order to fit the tree.

Tuning the algorithm to alternate between different algorithms to balance a subtree, eliminates this problem. This removes the penalty for calculating the median at unsuitable problem sizes.

Our current implementation only use one block per subtree, which means we are only able to use one SM to balance the subtree. By utilizing several blocks at the same time, big subtrees could be processed faster. However, since all blocks are used to balance one subtree, only one subtree can be balanced at a time.

The implementation details are omitted here, but can be found in Appendix~\ref{sec:radix_select}. In short, the implementation follows the outline of the radix-select implementation, but the thread synchronization is performed on the CPU, between kernel launches. This enables us to communicate between the different blocks.

We also would like to improve performance, when Balance-Subtree is applied to many small subtrees. For instance, at level $18$ there are $131072$ different subtask with only an average size of $7$. The previous implementation divided all these subtask between a small number of SM, typically $8-32$ on current NVIDIA GPUs\@. The algorithm the uses to many cores to balance a subtree of only $7$ elements, which is not a efficient way to divide resources.

Letting just one thread handle the Balance-Subtree operation for small subtrees, would let us process more subtrees in parallel, improving performance. With this parallelization, each thread is responsible for it's own subtree, and communication with other threads is no longer needed. With the need for communication eliminated, we can utilize Algorithm~\ref{alg:seriel_tree_build}. 

\begin{figure}[ht!]
\centering
\includegraphics[width=100mm]{../gfx/the_jumps_final.png}
\caption{Visualization of the final improvments on the k-d tree build implementation.}
\label{fig:the_jumps_final}
\end{figure}

Now that a lot of improvements have been made, lets take a look at the results. Figure~\ref{fig:the_jumps_final} compare our intermediate result with our new and improved version, and the changes made a huge impact on the performance. The jumps disappeared, and was replaced by a faster and smoother curve, indicating that the CUDA resources is perfectly balanced. The final implementation can be found in Appendix~\ref{sec:parallel_quick_select}.
% subsection further_development (end)

% section development_of_a_parallel_kd_tree_build_algorithm (end)

\section{Development of a parallel kd-tree search algorithm} % (fold)
\label{sub:development_of_a_parallel_kd_tree_search_algorithm}

\section{The quest for a fast KNN search} % (fold)
\label{sec:the_quest_for_a_fast_KNN_search}

This document is a summary of our most recent (7 February 2014) findings, in the quest for a fast kNN search algorithm. The most up to date information can be found in the
%[release notes](https://github.com/hgranlund/tsi-gpgpu/tree/master/src/kNN#v14-release-notes)
for the most recent version of this project.

Our initial investigation led us to believe that a serial implementation could be as fast as the parallel brute-force solution, for point clouds with fewer than 1 000 000 points, given that both algorithms start with an unordered set of points. Reimplementing the brute-force algorithm with bitonic sort, and optimizing for three dimensions, has shown us that this initial belief was unsupported, and currently the brute force algorithm is faster when starting from a unorganized set of points. When considering repeated querying of the same point cloud, the k-d tree based solution pulls ahead, as most of its running time is spent building the k-d tree for querying. If building the k-d tree could be parallelized this could change. although documented in literature, such an parallelization is still elusive.

In order to make the document more readable, we have included short descriptions of the algorithms used, a short reference to theoretical time complexity. We then go on to list our current results, problematic areas and possible improvements.

The following papers, available in the resources folder, forms the literary basis for our current work.

Related to the brute force approach:
\begin{itemize}
    \item Improving the k-Nearest Neighbor Algorithm with CUDA - Graham Nolan
    \item Fast k Nearest Neighbor Search using GPU - Garcia et al.
    \item K-nearest neighbor search: fast gpu-based implementations and application to high-dimensional feature matching - Garcia et al.
\end{itemize}

Related to the k-d tree based approach:
\begin{itemize}
    \item Real-Time KD-Tree Construction on Graphics Hardware - Kun Zhou et al.
\end{itemize}


\section{Brute force based effort} % (fold)
\label{sub:brute_force_based_effort}

\subsection{Garcia's base algorithm} % (fold)
\label{ssub:garcias_base_algorithme}

Garcia's algorithm is based on a naive brute-force approach. It consists if two steps:
\begin{enumerate}
    \item Calculate the distance between all reference points and query points.
    \item Sort the distances and pick the k smallest distances.
\end{enumerate}

Garcias implementation supports any number of dimensions, reference points and query points (or up to ~65000, number of blocks in the GPU). Due to this feature the algorithm use a lot of extra computation power when only one query point and a small dimensions is selected.

Time complexity:

Steps:

\begin{enumerate}
    \item O(n). Every reference point must be evaluated once. Since all calculations are independent, we have a large potential for parallelizing.
    \item Insertion sort: O(pow(n, 2)).
\end{enumerate}
% subsection garcias_base_algorithme (end)

\subsection{Our reimplementation} % (fold)
\label{ssub:our_reimplementation_2}

Bitonic-sort:

Graham Nolan discusses the possibility of improving Garcia's algorithm by reimplementing step two with a bitonic sort. His source code has not been available to us, but he states that the run-time improvements was significant. As well as choosing bitonic sort for the sorting stage of our algorithm, our implementation supports up to 15 000 000 points before memory errors occur, and we have limited the number of dimensions to three.

Time complexity:

Steps:
\begin{enumerate}
    \item O(n).
    \item Bitonic sort: worst case = O(n * log2(n)), average time ( parallel) = O(log2(n)).
\end{enumerate}

Results:

Testing the different algorithms for a range of point cloud sizes and a fixed value for k, gave the following results.

\begin{figure}[ht!]
\centering
\includegraphics[width=120mm]{../gfx/knn-brute-force-vs-serial-k-d-tree.png}

\caption{knn-brute-force-vs-serial-k-d-tree.png}
\label{fig:knn_brute_force_vs_serial_k_d_tree}
\end{figure}

We see that our reimplementation of the brute-force algorithm performs well overall, notably improving on Garcia's implementation (only visible as a short line in the beginning of the graph, due to the restricted number of points it is able to compute). Still more speed is desired before good interactive usage can be achieved.

Test with n = 8 388 608:

\begin{itemize}
    \item Memory transfer 21.1 ms
    \item Calculate all distances 2.5 ms
    \item Bitonic sort 176 ms
    \item Total 200 ms
\end{itemize}

Min-Reduce:

An other possibility to improve step 2 is to use a reduce operation to get the smallest distances. This can be done k times to get the k smallest values.

Time complexity:

Steps:

\begin{enumerate}
    \item O(n).
    \item Min-reduce: k* log2(n)).
\end{enumerate}

Memory optimalisation:

We have done some memory optimization based on a
%[presentation](https://github.com/hgranlund/tsi-gpgpu/blob/master/resources/kNN/reduction.pdf)
from Nvidia.

The optimizations include:

\begin{itemize}
    \item Shared memory utilization.
    \item Sequential Addressing.
    \item Complete for-loop Unrolling.
\end{itemize}

\begin{figure}[ht!]
\centering
\includegraphics[width=120mm]{../gfx/knn-brute-force-reduce-memory-opt.png}

\caption{Comparison between knn-brute-force-reduce with and without memory optimizations (k=10).}
\label{fig:knn_brute_force_reduce_memory_opt}
\end{figure}

Results:

Test results of n = 8 388 608 with no memory optimization:
\begin{itemize}
    \item Memory transfer:  21.1 ms.
    \item Calculate all distances: 2.5 ms
    \item One min-reduce step : 4.8 ms.
    \item Total time: (23.7 + k*4.8) ms.
\end{itemize}

Test results of n = 8 388 608 with memory optimization:
\begin{itemize}
    \item Memory transfer:  21.1 ms.
    \item Calculate all distances: 2.5 ms
    \item One min-reduce step : 1.7 ms.
    \item Total time: (23.7 + k*1.7) ms.
\end{itemize}

\begin{figure}[ht!]
\centering
\includegraphics[width=120mm]{../gfx/BitonicVSreduce.png}

\caption{Comparison between bitonic and reduce (k=10).}
\label{fig:bitonic_vs_reduce}
\end{figure}

Possible improvements:

\begin{itemize}
    \item Memory improvements. Use shared memory and texture memory.
    \item Modify bitonic sort, so do not need to sort all points. We can split the distance array to fit into the GPU blocks, move the smallest values in each block, then sort the moved values. ~O((n/b)* b*log2(b)) subsetof O(n/b), b = Number of threads in each block, n= number of reference points
    \item Replace bitonic sort with min reduce. O(k*log2(n)).
\end{itemize}
% subsection our_reimplementation (end)
% section brute_force_based_effort (end)

% \section{KD-tree based effort} % (fold)
% \label{sub:kd_tree_based_effort}

% A k-d tree can be thought of as a binary search tree for graphical data. A few different variations exist, but we will focus our explanation around a 2D example, storing point data in all nodes. The plane is split into two sub-planes along one of the axis (in our example the y-axis) and all the nodes are sorted as to whether they belong to the left or right of this split. To determine the left and right child of the root node, the two sub-planes are again split at an arbitrary point, this time cycling to the next axis (in our example the x-axis) and the

% \begin{figure}[ht!]
% \centering
% \includegraphics[width=120mm]{../gfx/Kdtree_2d.png}

% \caption{2D KD-tree}
% \label{fig:kdtree_2d}
% \end{figure}

% In order to build a k-d tree for 3D space, you simply cycle through the three dimensions, instead of two.

% Given the previous splits and selection of nodes, the resulting binary tree would be as shown in the illustration under. (All illustrations gratuitously borrowed from
% %[Wikipedia](http://en.wikipedia.org/wiki/k-d_tree))

% \begin{figure}[ht!]
% \centering
% \includegraphics[width=120mm]{../gfx/Tree_0001.png}

% \caption{corresponding-binary-tree}
% \label{fig:tree_0001}
% \end{figure}

% Given that the resulting binary tree is balanced, we get an average search time for the closest neighbor in O(log2(n)) time. For values of k << n, the same average search time can be achieved, with minimal changes to the algorithm, when searching for the k closest neighbors. It is known from literature that balancing the tree can be achieved by always splitting on the meridian node. Building a k-d tree in this manner takes O(kn log2(n)) time.

% Interested readers is encouraged to look at the paper Multidimensional binary search trees used for associative searching by Jon Louis Bentley, where k-d trees first was described.

% \subsection{The serial base algorithm} % (fold)
% \label{ssub:the_serial_base_algorithm}

% \begin{enumerate}
%     \item Build a balanced k-d tree from the point cloud.
%     \item Query the tree for different sets of neighbors.
% \end{enumerate}

% Time complexity:

% Steps:
% \begin{enumerate}
%     \item O(n log2 n). Achieving this speed is dependent on an efficient algorithm for finding the meridian.
%     \item Approximately O(log2 n), but dependent on size of k.
% \end{enumerate}

% \begin{figure}[ht!]
% \centering
% \includegraphics[width=120mm]{../gfx/serial-k-d-tree-breakdown.png}

% \caption{serial-k-d-tree-breakdown}
% \label{fig:serial_kd_tree_breakdown}
% \end{figure}

% Results:

% As expected, almost all the time is spent building the tree. Querying for the closest neighbor in the largest tree took less than 0.0015 ms, but 9 seconds is a long time to wait for the tree to build.

% The paper Real-Time KD-Tree Construction on Graphics Hardware - Kun Zhou et al. offers interesting, although slightly complex, ideas to an efficient parallelization of k-d tree construction. I order to save time, a good amount of time was spent searching for, and trying out, different open source implementations based on this paper. This search was unsuccessful. All the implementations we managed to find was problematic due to lack of updates, often not updated since 2011, and still running on CUDA 4.1, lack of documentation, lack of generalization or dubious source code.

% A more uplifting find was several references to Real-Time KD-Tree Construction on Graphics Hardware in material published by Nvidia, regarding their proprietary systems for ray tracing. A graphics rendering technique often reliant on k-d trees, and indeed dependent on high performance.
% % subsection the_serial_base_algorithm (end)

% \subsection{Our reimplementation} % (fold)
% \label{ssub:our_reimplementation}

% Finding a way to represent the kd-tree boils down to representing a binary tree. This can be done in several ways, but we had some criteria for our representation:

% \begin{itemize}
%     \item The kd-tree should be memory-efficient, at best explicitly storing only the actual point coordinates. This since we want to store the largest possible amount of points on the smaller memory of the GPU.
%     \item The kd-tree representation should be easy to split, distribute and join, since we want to build it on several independent processes.
% \end{itemize}

% After a bit of research and trial and error, we choose to represent the kd-tree as a array of structs, representing points, with an x, y and z coordinate stored in an short array. In order to turn this array into a binary tree, the following scheme was derived. Given an array, the root node of that array is the midmost element (the leftmost of the two, given an even number of elements). To find the left child of the root, you simply find the midmost element of the left sub-array, and likewise for the right child. This representation have some advantages and drawbacks.

% Advantages:
% \begin{itemize}
%     \item Can represent binary trees where not all leaf-nodes are present. This is the minimal requirement for representing perfectly balanced k-d trees.
%     \item Joining or splitting subtrees is as simple as appending or splitting arrays.
%     \item Minimal memory overhead, k-d trees can even be built in-place on the array.
% \end{itemize}

% Drawbacks:
% \begin{itemize}
%     \item Cannot represent imperfectly balanced k-d trees. This mens that the median cannot be calculated through heuristic methods, but enforces a tree optimized for fast queries.
%     \item Location of children and parents have to be recalculated for all basic traversing of the tree, with may reduce the performance of queries on the tree. In order to eliminate this drawback, a index cache is computed before the search is performed.
% \end{itemize}

% Given this representation of the kd-tree, the base kd-tree building algorithm can be expanded to the following:

% Steps:
% \begin{enumerate}
%     \item Find the exact median (of the current split dimension) in this sub-array, and swap it to the midmost place.
%     \item Go through all elements in the list, and swap elements, such that all elements lower than the median is placed to the left, and all elements higher than the median is placed to the right.
%     \item Repeat for the new sub-arrays, until the entire tree is built.
% \end{enumerate}

% Results:

% The serial kd-tree build times was similar or better than the base algorithm, so it is not discussed here

% \begin{figure}[ht!]
% \centering
% \includegraphics[width=120mm]{../gfx/awg-query-time-old-vs-new.png}

% \caption{Awg query time old vs new}
% \label{fig:awg_query_time_old_vs_new}
% \end{figure}

% The graph shows that our serial implementation of search in this data structure, given a pre-calculated index cache, is as fast, or slightly faster than the base algorithm. The possibility of improving the search even more, by storing all calculated distances in a distance cache was also explored, but we can see from the graph that the overhead associated with this operation did outweigh the benefits.

% Some instability is apparent in the graph, but this is probably due to the author running other programs in the background when performing the test.

% \begin{figure}[ht!]
% \centering
% \includegraphics[width=120mm]{../gfx/n-query-time-old-vs-new.png}

% \caption{n query time old vs new}
% \label{fig:n_query_time_old_vs_new}
% \end{figure}

% The trend is exaggerated when considering N searches. Searching with an index cache is overall fastest, and gives a total search time of ~17 s for 14 million searches in a point cloud of 14 million.


% Possible improvements:
% \begin{itemize}
%     \item Run several queries in parallel. Easy to implement, as parallelization is trivial due to independent queries, but is dependent on efficient transfer and storage of the kd-tree on the GPU.
%     \item Explore performance on a variable number of k.
% \end{itemize}
% % subsection our_reimplementation (end)



% % subsection parallel_improvements (end)

% \subsection{V1.4 Release notes} % (fold)
% \label{ssub:v14_release_notes}

% Version 1.4 introduces the possibility of a variable k when searching. Testing the impact of varying the size of k was performed with a fixed number of 1000 repeated single queries. The timing results are shown in the following graph.

% \begin{figure}[ht!]
% \centering
% \includegraphics[width=120mm]{../gfx/v14_variable_k.png}

% \caption{Variable k timing results.}
% \label{fig:v14_variable_k}
% \end{figure}

% As expected, increasing k seems to increases the runtime with a constant factor. How this will affect searches in bigger trees and with a larger number of query-points should be explored next.

% Timing tests of the build time and query time for n queries and k = 1 was performed on a GeForce GTX 560 and at Amazon Web Services (AWS). For the GTX card we get the following graph.

% \begin{figure}[ht!]
% \centering
% \includegraphics[width=120mm]{../gfx/v14_build_query_gtx.png}

% \caption{Build and query times for v1.4.}
% \label{fig:v14_build_query_gtx}
% \end{figure}

% Querying for n points still takes a lot of time, but the time increase related to the number of points seems to be lower than the build time. This trend is more prominent when we look at the results from our tests on AWS, graphed below.

% \begin{figure}[ht!]
% \centering
% \includegraphics[width=120mm]{../gfx/v14_build_query_aws.png}

% \caption{Build times and queries on AWS.}
% \label{fig:v14_build_query_aws}
% \end{figure}

% Here we see that the runtime for the tree building algorithm catches up with the search time, and surpasses it around 50 million points. This is an interesting result.

% From this graph we can see that on the AWS GPU, we are able, for 90 million points, to build the tree and query for the closest point, k = 1, in a total of ~30s + ~26s = ~56s < one minute.

% Source data for all graphs can be found in
% %[this spreadsheet](https://docs.google.com/spreadsheets/d/1I-qxnPa2FuYs7ePQC7d9v0GVHoYlr4CY6QdJbbNUlYo/edit?usp=sharing).

% % subsection v14_release_notes (end)

% \subsection{V1.3 Release notes} % (fold)
% \label{ssub:v13_release_notes}

% Version 1.3 introduces a couple of new features. Firstly the tree-building algorithm has been updated to also cache the location of the different children of each node in the tree. A small bug related to partition of the point-cloud list was also ironed out. This gives a tree building algorithm whit the following runtime results, comparable with the previous versions of this implementation:

% \begin{figure}[ht!]
% \centering
% \includegraphics[width=120mm]{../gfx/construction_v13_gtx_560.png}

% \caption{Construction timing results v13 @ gtx 560.}
% \label{fig:construction_v13_gtx_560}
% \end{figure}

% After a surprising amount of fiddling, querying for a large number of query-points was finally parallelized in version 1.3. This gave improved performance when querying many times in the same point cloud. 14 million queries in a point cloud of 14 million points can now be done on average in ~8 seconds. Compared to the estimated ~17 seconds needed by the serial implementation.

% \begin{figure}[ht!]
% \centering
% \includegraphics[width=120mm]{../gfx/n_queries_v13_gtx_560.png}

% \caption{n queries timing results v13 @ gtx 560.}
% \label{fig:n_queries_v13_gtx_560}
% \end{figure}

% Unfortunately, this early parallelization gave us quite unstable results. The data in the graph is based on ten timing runs, where the average, minimum and maximum values are collected in three series. As we can see, there is quite a big spread between the best and worst results. Looking at the raw data points for the ten series confirms that this is not a problem caused by a small number of outliers:

% \begin{figure}[ht!]
% \centering
% \includegraphics[width=120mm]{../gfx/scatter_n_queries_v13_gtx_560.png}

% \caption{Scatter n queries timing results v13 2 gtx 560.}
% \label{fig:scatter_n_queries_v13_gtx_560}
% \end{figure}

% Some instability would be expected, as the amount of pruning that can be achieved when searching for points in the kd-tree is dependent on the value you search for, but this amount of spread was unexpected. This behavior should be investigated further, as it seems to be indicating some kind of implementation error.

% Combining the results from the search and the tree-building, gives the following runtime for a sequence of building and N queries:

% \begin{figure}[ht!]
% \centering
% \includegraphics[width=120mm]{../gfx/constructionand_n_queries_v13_gtx_560.png}

% \caption{Construction and n queries v13 @ gtx 560.}
% \label{fig:constructionand_n_queries_v13_gtx_560}
% \end{figure}
% % subsection v13_release_notes (end)
% % section kd_tree_based_effort (end)


\cleardoublepage
