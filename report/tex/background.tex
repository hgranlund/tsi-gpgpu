%!TEX root = ./main.tex
\chapter{Background}

The purpose of this chapter is to cover some background material for the research presented in this thesis. A introduction to the kNN problem is given, and its relation to all-kNN, and q-kNN is discussed. In addition, a presentation of relevant parallel programming principles, programming practices and tools is given.

\section{A short introduction to kNN search problem} % (fold)
\label{a_short_introduction_to_kNN_search_problem}

The k-nearest neighbors (kNN) search problem, is, in short, the problem of locating the k closest neighbors to a given query point, with k being some positive natural number. This is intuitively a quite simple problem to grasp, and for most of our purposes an intuitive understanding of the problem will be sufficient, but let us start by looking a bit closer at the properties of kNN search in general.

If we consider $p$ to be a point in d-dimensional space so that $p = [x_1, x_2,\dots, x_d]$. Given a set of size $m$, consisting of such points $S = [p_1, p_2,\dots, p_m]$, a additional query point $q$, also in n-dimensional space, and some distance metric $D = f(p, q)$, the kNN problem is to find k points in $S$, such that the sum of distances in relation to $q$ is minimized. We introduce the term reference points to be the set $S$, to differentiate it from the query points $Q$. 

It is worth to note that since we are not limited, either in the number of dimensions, or distance metric we choose, the kNN problem is applicable in many different situations. If we e.g.\ wanted to construct a system for finding beer with similar flavor to one beer we just tasted, we could build a database of different beers, categorized by flavor dimensions like bitterness, maltiness, sweetness, and so on. Then, to find beers similar in flavor to the one we just tasted, we would perform a kNN query on this database, using a suitable distance metric, and the beer we just tasted as our query point.

The general nature of the kNN problem makes it relevant in many research and industrial settings, from 3-d object rendering, content based image retrial, statistics and biology \citep[Introduction]{Garcia2010}.

When wanting to query point cloud data, we can make some simplifications to this general kNN problem. In our research we are only concerned with three spacial dimensions, and their Euclidean relations. These restrictions does not cover all the possible values that might be interesting to consider when applying a kNN algorithm to point cloud data. The color value of each point could e.g.\ also be interesting to include. But for most applications, TSI application included, three dimensions, and an Euclidean distance metric is all we need. Throughout this text we will use the term kNN to refer to this simplification of the kNN problem.

When studying points clouds, it can be interesting to study the closest points to all points in the point cloud. In order to compute this, you would simply perform kNN queries, using all the points in the point cloud as query points. In order to refer to this variant of the kNN search problem, we will use all-kNN\@.

Another similar variant is application of the kNN algorithm to a set of query points of size q. In this version of the problem, you are not limited to the query point set being the points in the point cloud, it can be any set of three dimensional points. We will refer to this problem variant as Q-kNN, and note that all-kNN is a subproblem of Q-kNN\@.
% section a_short_introduction_to_kNN_search_problem (end)

\section{A short introduction to relevant parallel programming principles} % (fold)
\label{sub:a_short_introduction_to_relevant_parallel_programming_principles}

Parallel programming is programming targeting parallel computing, where sections of the program is executed simultaneously on multiple cores, processors, machines, or other suitable environments. We use the term parallelization to mean transformation of computational instructions intended for sequential execution to simultaneous, or parallel, execution.

Parallelization can be introduced on several different levels in a computer, with bit-level parallelization, where bit level operations is parallelized within a processor as is the case for 64-bit, 32-bit, 16-bit, etc. processors, being a common low level form. In order to avoid confusion, this thesis is not concerned with such levels of parallelization, but will instead focus on higher level parallelization, related to developing and implementing algorithms in a regular programming language. We will also use the term parallel unit as a general term for a single computational unit in a theoretical parallel machine, with a unlimited number of parallel units.

It is easy to grasp that parallelization can speed up the execution of a program. Given a problem where we want to add a one to every element in a list of numbers. In a sequential program we could go through the list of numbers, adding one to each element in each step. This would roughly take the time needed for adding one number to another, times the number of elements in the list. In out theoretical parallel machine, we can simply assign all the elements in the list to a different parallel unit and ask every unit to add one to its assigned number. This would take much less time than the sequential algorithm, just the time needed to add a number, since all numbers in the list is added at the same time.

Adding one to the elements of a list is a trivial problem, and unfortunately not all problems can be parallelized as easy as this. Even simple problems can be hard to parallelize. Consider adding all the numbers together, instead of adding one to each one. This is a very similar problem, we still only has to add a number to all of the elements in the list. The problem is that we does not know what to add to a given element before all the previous elements has been added together. We could calculate the sub-sum of small subsections of the list in parallel, and then add the resulting sub-sums together in a sequential fashion, but then a part of our program does not execute in parallel. For many problems, we cannot entirely parallelize the execution, because some data has to be transfered between the threads. We need a way to let the parallel units communicate with each other.

Communication is often a large factor in limiting the performance boost that can be harnessed from parallelization. It gives rise to implied sequential execution, since the parallel units have to wait for data from another unit. Communication is also often inherently slow, and carries a high overhead due to low transfer speeds. The possibility of minimizing the amount of communication is therefor a major factor in determining if a sequential algorithm can be successfully parallelized.

\subsection{Shared memory architecture} % (fold)
\label{sub:shared_memory_architecture}
In a parallel computer using shared memory architecture (SMA), the different parallel units all share a global shared memory. The parallel units usually are different processors, often located within the same physical chip like a multi-core CPU. 

This is the architecture used by most modern desktop computers. Different varieties exist, with separate processor cache for each processor core, shared cache or even a combination, with shared L2 cache and distributed L1 cache. From the point of view of this thesis, all these varieties would fall under the SMA classification.

Shared memory is easy to work with and understand, and communication between different parallel units can be facilitated quite easily, and relatively fast, by reading and writing the same memory. The drawback is that the programmer must ensure that the different parallel units does not try to access the same memory at the same time or in the wrong order. SMA still works well for smaller parallel systems.
% subsection shared_memory_architecture (end)

\subsection{Distributed memory architecture} % (fold)
\label{sub:distributed_memory_architecture}
In a computer system using distributed memory architecture (DMA), each parallel unit contains both processor and memory. The processor can only access data from its own local memory, and if data is needed from another parallel unit, it has to be transfered from that units local memory into the memory of the unit in need of the data. DMA computer systems usually scales better, considering the number of parallel units, since one memory does not have to facilitate all parallel units. The drawback is that communication carries a higher overhead, since data has to be transferred between the different local memories of each unit.

Computer systems using a distributed memory architecture does often more closely resemble many individual computers, working in parallel on the same problem, and communicating using specialized networking components.

This is the architecture favored in todays supercomputers. Many varieties exist, especially concerning the network layout between the different machines. Since each individual parallel unit in such systems usually can be considered to be an individual computer, each unit often has a internal shared memory architecture, like desktop computer. This makes the entire system capable of harnessing the strength of both SMA and DMA, but increases the complexity that the programmer has to handle. As we will discover later in the text, GPUs are organized using an architecture with this kind of hybrid architecture.
% subsection distributed_memory_architecture (end)

\subsection{Parallel speedup} % (fold)
\label{sub:parallel_speedup}
Parallel speedup, or just speedup as it is often called, is a measure of how much faster a parallel algorithm is compared to its sequential counterpart.

Let $T_s$ be the execution time of the sequential algorithm, and $T_p$ be the execution time of the parallel algorithm on a system with $p$ parallel units. The speedup $S_p$ is then defined as $S_p = T_s / T_p$.

In the ideal situation, the relation between the speedup and the number of parallel units will be linear. Due to overhead related to possible communication, and the use of a more complex framework for the parallel code, this is usually not possible to achieve. We therefore have the parallel efficiency metric $E_p = S_p / p$, which describes how much is lost to such factors.

Speedup and efficiency can be a good measure of how well the algorithm is parallelized, but it can not necessarily be used to determine if one parallel algorithm is faster than another. Parallelizing many inefficient brute-force algorithms can be done with excellent speedup and efficiency, but the resulting algorithm will often be considerably slower than the parallel version of a better sequential algorithm, or even just a better sequential algorithm. Parallelizing bad code will result in just as bad code, and thorough study of efficient algorithms should always be carried out before attempting any parallelization.
% subsection parallel_speedup (end)

\subsection{The APOD design cycle} % (fold)
\label{sub:the_apod_design_cycle}
In order to work efficiently with parallelization problems, many programmers adopt a work-flow similar to the APOD design cycle\citep[Assess, Parallelize, Optimize, Deploy]{cuda_c_best_practices_guide}. The cycle consists of four steps:

\begin{enumerate}
    \item \textbf{Assess:} Locate the parts of the code which take up most of the run-time. Re-writing an entire application for parallel execution is usually very hard and time consuming, if it is even possible. The best results for the user might be to just parallelize the one algorithm that is taking a long time to execute. Parallelization might not even be the answer, a faster sequential algorithm could also be a solution or part of the solution.

    \item \textbf{Parallelize:} Investigate if any pre-written library of parallel functions could be used as part of, or the entire solution. Try to pinpoint which part of the code that can execute in parallel, and which part is dependent on communication. If the code is inherently dependent on communication, sequential alternatives, more suitable for parallelization, might be researched. Just not be tempted to use a inefficient algorithm because it is easy to parallelize.

    \item \textbf{Optimize:} Dependent on the parallel computer system, programming language, and other tools you are using, a number of conventional optimization strategies might be applied. This should not be forgotten when writing parallel code.

    \item \textbf{Deploy:} Run your application on real hardware, test thoroughly, and compare the results to the original. Did the parallelization actually increase the performance of the application? Deploy the code to potential users. They will benefit from the increased performance, and you will get feedback if any bugs exist.
\end{enumerate}
% subsection the_apod_design_cycle (end)

% section a_short_introduction_to_relevant_parallel_programming_principles (end)

\section{A short introduction to GPU programming and CUDA} % (fold)
\label{sub:a_short_introduction_to_gpu_programming_and_cuda}

Moore's law has been a gift to all programmers the past 50 years. The law predicts that performance of integrated circlets would double every two years. Resulting in automatically faster algorithms without any reimplementation. However, since the so called Power Wall in 2002, the world has been changing. The performance boost  in processors, and verification of Moore's law, is no longer limited to a single processor core, but to multiple cores. As a result, all programs and algorithms has to be rewritten to follow the multi-core evolution.

Since the Power Wall a lot of work have been done regarding parallel programming tools, like OpenMP, CUDA, MPI and OpenCl. These tool have a varied types of supported architectures. OpenMP supports a shared memory architecture, where all cores have access to the same memory. It is an implementation of multi threading, whereby a master thread divide the workload to different forked slave threads. MPI (Message Passing Interface) is a library specification for message passing, and is proposed as a standard by a broadly based committee of vendors, implementors and users. It is mostly used for communication between processors running in a distributed memory system, as most supercomputer clusters are designed today. MPI is therefore the most used parallel computer method used on supercomputer clusters today.
% An application build on both OpenMp and MPI are called a hybrid. They are used on systems that use distributed memory across different nodes, where each node uses a shared memory architecture.  These applications use MPI to send data to each node and OpenMP on individually nodes.

Although supercomputers are fast and can have thousands of cores, they are highly expensive and are not accessible for normal consumers. Out of the current consumer level products, GPU's represent the most extreme in parallelized hardware. NVIDIA GeForce GTX 000, for example, can execute 23,040 threads in parallel, and in practice requires at least 15,000 threads to reach full performance \citep{karras2012}. The reason GPU's have such a massive amount of parallelized threads is that each thread is very lightweight. The benefit is that together they can achieve extremely high instruction throughput. This makes the GPU perfect for high performance computing.

\subsection{General-purpose computing on graphics processing units} % (fold)
\label{ssub:general_purpose_computing_on_graphics_processing_units}

The GPU's extreme properties, and the highly increasing interest in multi core systems, have also made the General-purpose computing on graphics processing units (GPGPU) programming landscape rapidly evolve. GPGPU is the utilization of GPU in applications, to perform heavy computation, that is normally handled by the CPU\@. The preveous mentioned parallel programming tools, OpenMp and MPI, are not  directely desiged for GPU programming, and several new programming approaches have appeared. Two widely used approaches are Compute Unified Device Architecture (CUDA) and The Open Compute Language (OpenCL).

OpenCl is a low-level framework for heterogeneous computing for both CPU and GPU's. It includes a programming language, based on C99, for writing kernels. Kernels are methods that executes on the GPU\@. It also includes an API that are used to define and control the platform.

In contrast to the open OpenCL, the dominant proprietary framework CUDA\@, is only desiged for GPU programming. It is, as OpenCL, based on a programming language and a programming interface. Science CUDA is created by the vendor, it is developed in close proximity with the hardware.

For a deep and thoroughly survey of GPGPU programming, techniques and applications take a look at John D. Owens article from 2007 \citep{Owens:2007:ASO}.
% subsection general_purpose_computing_on_graphics_processing_units (end)

\subsection{NVIDIA GPU architecture} % (fold)
\label{ssub:nvidia_gpu_architecture}

The biggest difference between GPU and CPU's are there main objective. The CPU is optimized for low latency, while the GPU is optimized for high throughput. For a chip to achieve low latency it needs to have ha huge amount of cache available, which makes it harder to fit a lot of cores. To get high throughout a lot of cores or ALUs are needed. The GPU has therefore a small control unit and cache, but is packed with cores. Basically we can say that the difference is how the transistors are used. The GPU has more transistors dedicated to computation, then the CPU\@.

At the hardware level, a NVIDIA GPU is build around a scalable array of multithreaded Streaming Multiprocessors (SMs). A multiprocessor are designed to execute hundreds of threads concurrently. It is organized according to an architecture called SIMT (Single-Instruction, Multiple-Thread), where each SMs creates, manages, schedules and executes parallel threads in groups of 32. This is called a warp. Threads composing in a warp starts at the same program address, but they have their own register state and instruction address, and is therefore free to branch and execute independently \citep{cuda_programming_guide}.

An other impotent part of the GPU architecture is the memory hierarchy. Global memory is bottom-most part of this hierarchy and is analogous to RAM on the CPU\@. The global memory can be used to transfer memory to and from the host CPU\@. Each SM contains a fast shared memory, which is shared across each thread in the SM\@. The threads also have their own 32-bit registers. There also exists other types of memory, called constant memory and texture memory. Compared to the CPU, the peak floating-point capability and memory bandwidth of the GPU, is an order of magnitude higher \citep{Liangcu}.


% More about hiding latency: http://docs.nvidia.com/cuda/cuda-c-programming-guide/#multiprocessor-level
% Maybe talk more about memory heiartghy, on-chip and of chio,
% Local memory:
 % Does not physically exist. It is an abstraction to the local scope of a thread.
% Actually put in global memory by the compiler.


% subsection nvidia_gpu_architecture (end)

\subsection{CUDA programming model} % (fold)
\label{ssub:cuda_programming_model}

The CUDA programming model is designed to make the heterogeneous GPU+CPU programming easier. The GPU works as a computation accelerator for the CPU, and the programming model should therefore be an easy bridge between the two. CUDA have created this bridge based on a runtime library, compiler and C language extensions. The C language extensions enables the programmer to create and launch methods on the GPU, through the runtime library. These methods are called kernels.

A CUDA program is is based on a data-parallel behavior, where threads is running in parallel. The execution of a kernel is organized as a grid of blocks consisting of threads. When a kernel grid is launched, on the host CPU, blocks of the grid is enumerated and distributed among the SMs. The blocks then executes concurrently on individual SMs. As blocks terminates, new blocks are added. Only threads in a block can communicate with each other, by creating synchronization walls. They can also communicate through the fast shared memory, that is located on each individual SM\@. Each block and thread have their own id, which often is used to determines what portion of data the thread should process.

% Maybe we should write about have to optimize parallel programs:
% 1. find a good serial algorithm
% 2. find a good parallelization strategy
% 3. find a good parallel algorithm
% 4. Small improvements like:
%     * do memory optimization (types of memory)
%     *. minimize data divergence
%     *. optimize arithmetic operations.

% subsection cuda_programming_model (end)
\cleardoublepage
